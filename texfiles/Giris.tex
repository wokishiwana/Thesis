\chapter{GİRİŞ VE AMAÇ}
\label{girisveamac}

Geçmişteki motor sistemleri basit ve verimsiz güç kaynaklarıydı. Gelişen malzeme bilimi daha dayanıklı motor parçaları üretmeye, elektronik sistemler daha sağlam motor kontrol üniteleri yaratmaya
ve bilgisayarlar ise bu sistemleri fiziksel bir ortama ihtiyaç duymadan bir araya getirecek şekilde günümüzde yerini aldı. 19. yy.da ortaya çıkan güçsüz içten yanmalı motorlar veya küçük
ve zayıf elektrik motoru içeren taşıtlar yerlerini birden fazla sayıda ve türde motor içeren kompleks yapılara bıraktı. Bu gelişmeler beraberinde daha farklı ve daha sağlam kontrol ihtiyacı da 
getirdi. Mekanik, tasarım aşamasında üretilip tasarımı dondurulan yakıt, hava, ateşleme, güç aktarımı vb. sistemler elektronik, ortam değişkenlerine cevap vermesi gereken 
dinamik karar ortamlarına dönüştü. 

Kontrol algoritmalarında yeni ve yüksek sayıda değişkeni gözlemleyerek tasarım doğrultusunda sistemi yönlendirecek algoritmalara ihtiyaç doğdu. Çünkü tek tür motorlu sistemler yerine
hibrit sistemler geldiğinde birden fazla güç kaynağı ve buna bağlı olarak farklı kombinasyonlar hizmet vermeye başladı. Önceden, sadece içten yanmalı motor içeren sistemlerin menzilleri
yakıt deposu bitene kadar taşıtın çalıştırılmasıyla hesaplanabilecekken, hibrit sistemlerin getirdiği ikincil güç üretim kaynağı ve bu kaynağın kullandığı enerjiyi depolayan sistem, örneğin
elektrik motoru ve bağlı olduğu batarya, taşıtların menzillerini nihai optimizasyon parametresi olarak kullanan kontrolcü ve algoritmalarda değişikliğe sebep oldu.

Kontrol algoritmalarının yanında, hibrit motor sistemlerinin de birtakım avantajları vardır. Hibrit motor sistemlerinin yer aldığı taşıtların geleneksel taşıtlara göre
avantajlarının bazıları daha iyi yakıt ekonomisi, daha az emisyon, taşıtın yavaşlaması için kaybedilen kinetik enerjinin bir kısmını yeniden kazanması, sadece elektrikli motorun veya
sadece içten yanmalı motorun yer aldığı taşıtlara göre daha yüksek menzil, farklı yakıtlar ve farklı motorlar kullanabilme imkanıdır (\cite{advofhybrids}).

