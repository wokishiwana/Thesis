%Özette tez çalışmasının amacı, kapsamı, kullanılan yöntemler ve varılan sonuçlar açık ve öz olarak belirtilmeli, bunlar alt başlıklar altında sunulmamalıdır.  Özetin uzunluğu 250 kelimeyi geçmemelidir. 

%Summary sayfasının içeriği ve düzeni tümüyle Özet sayfasının aynı olmalı ve (vii) ile numaralanmalıdır.

%Özet ve Summary’nin altına anahtar kelimeler/keywords yazılmalıdır.  Konu literatürde hangi kelimelerle geçiyorsa anahtar kelime olarak bu kelimeler kullanılmalıdır.

\chapter{ÖZET}

Bu tezde bir içten yanmalı motor bir de elektrik motorundan oluşan, 
güç ayrım cihazlı seri-paralel hibrit motor sisteminin kontrol algoritması modellenmiştir.
Modelleme MATLAB/Simulink platformunda yapılmıştır. Bu ortamın seçilme sebebi çeşitli kontrol kütüphanelerini
halihazırda içermesi ve özel kütüphaneleri oluşturarak saniye bazında simulasyon yapılabilmesidir.

Kontrol algoritması tasarlanırken batarya şarj durumunun belirli değerler arasında kalması amaçlanmıştır.
Bu sayede hem batarya sürekli optimum bölgelerde
tutulmuş hem de yüksek menzil elde edilmiştir. Kontrol algoritması dünyada yaygın olarak kabul edilmiş
farklı sürüş çevrimlerinde analiz edilmiştir. Farklı sürüş çevrimleri objektif kıyas ve
kontrolcü performansını incelemek için uygundur.
Analizler sonucunda kontrolcünün sağlamlığı, batarya şarj durumunun tüm sürüş çevrimlerinde istenilen bölgede kaldığı gösterilerek
kanıtlanmıştır. Bu çalışma gelecekteki çalışmalar ve kontrolcü algoritmaları için de yol göstermektedir. 

Anahtar kelimeler: Seri, paralel, hibrit, Simulink, kontrol, güç ayrım cihazı